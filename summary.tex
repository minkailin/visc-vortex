%\clearpage
\section{Summary and discussion}\label{summary}
We have performed customised hydrodynamic simulations of
non-axisymmetric instabilities in 3D viscous discs. We adopted
height-dependent kinematic viscosity profiles, such that the disc
midplane is of low viscosity ($\alpha\sim 10^{-4}$) and the disc
atmosphere is of high viscosity ($\alpha\sim 10^{-2}$). We were motivated  
by the question of whether or not the Rossby wave instability, and
subsequent vortex formation, operates in layered accretion discs.  

We first considered viscous disc equilibria with a radial density
bump and varied the vertical dependence of viscosity. 
This setup can isolate the effect of viscosity on the
linear RWI. We found that the linear RWI is unaffected by viscosity,
layered or not. The viscous RWI remains dynamical and lead to vortex
formation on timescales of a few 10s of orbits. We continued these
simulations into the non-linear regime, but found that vortices became
stronger as the viscous layer is increased in thickness. We suggest this
counter-intuitive result is an artifact of the chosen viscosity
profile because it is radially structured: viscosity attempts to
restore the equilibrium radial density bump, which favours the
RWI. This effect outweighs viscosity damping the linear instability. 

We also simulated vortex formation at planetary gap edges in layered
discs with a radially-smooth viscosity profile. %In this case viscosity
%only acts to damp the instability.  
Although vortex formation still 
occurs in layered discs, we found the vortex can be destroyed even when the viscous
layer only occupies the uppermost scale-height of the vertical domain
which is 3 scale-heights. This is significant because most of the disc
mass is contained within 2 scale-heights (i.e. the low viscosity
layer) but simulations show a viscous atmosphere inhibits long term
vortex survival.    
We found that the non-axisymmetric energy densities have
weak vertical dependence, so the disturbance evolves
two-dimensionally. It appears that applying a large viscosity in the
disc atmosphere is sufficient to damp the instability throughout the
vertical column of the fluid.

%We invoke the 3D vortex models described by \cite{barranco05} to
%explain why Rossby vortices are damped out even when viscosity is only
%applied in the disc atmosphere.  
\cite{barranco05} have described two 
3D vortex models: tall columnar vortices and short finite-height vortices. 
Rossby vortices are columnar, i.e. the associated vortex lines
extend vertically throughout the fluid column. One might have expected
an upper viscous layer to damp out vortex motion in the disc atmosphere, leading to a shorter
vortex. This, however, requires vortex lines to loop around the
vortex (the short vortex of \citeauthor{barranco05}). Such  
vortex loops form the surface of a torus \citep[see, for 
  example, Fig. 1 in][]{barranco05}, instead of ending on 
vertical boundaries. This implies significant vertical
motion near the vertical boundaries of the vortex, which would be  
difficult in our model because of viscous damping applied there. We
suspect this is why short/tall vortices fail to form/survive in our
layered disc-planet models. We conclude that vortex
survival at planetary gap edges require low viscosity
($\alpha\lesssim10^{-4}$) throughout the vertical extent of the
disc.     

%also because vorticity is not a function of height. horizontal vorticity
%diffusion at the disc atmosphere --> but can't have strong vorticity
%gradient w.r.t. z, so vorticity column diffuses altogether 

\subsection{Relation to other works}
\cite{pierens10} simulated the orbital migration of giant planets in
layered discs by prescribing a height-dependent viscosity profile. They
considered significant reduction in kinematic viscosity in going from
the disc atmosphere (the active zone, with $\alpha\sim10^{-2}$) to the
disc midplane (the dead zone, with $\alpha\sim10^{-7}$). According to
previous 2D simulations, such a low kinematic viscosity should lead to
the RWI \citep{valborro06,valborro07}. However, \cite{pierens10}
did not report vortex formation, nor are vortices visible from their
plots. {\bf Very recent MHD simulations of giant planets in a layered
  disc also did not yield vortex formation \citep{gressel13}. 
These} result {\bf are} consistent with our simulations. 

\cite{oishi09} carried out MHD shearing box simulations with 
a resistivity profile that varied with height to model a
layered disc: the disc atmosphere was MHD turbulent while the disc
midplane remained stable against the MRI. They envisioned the active
zone as a vorticity source for vortex formation in the midplane.  
Although their setup is fundamentally different to ours, they also
reported a lack of coherent vortices in the dead zone. They argued
that the MHD turbulence in the active layer was not sufficiently strong
to induce vortex formation in the dead zone. If MHD turbulence can be
represented by a viscosity, the lack of tall columnar vortices in
\cite{oishi09} is consistent with our results. That is,
even when MRI turbulence is only present in the disc atmosphere it is
able to damp out columnar vortices.     

\subsection{Caveats and outlooks}\label{caveats}
%We stress that the goal of the present study is to demonstrate the
%potential impact of disc vertical structure on Rossby vortices. We
%adopted simple 

The most important caveat of the current model is the viscous
prescription to mimic MRI turbulence. In doing so, an implicit
averaging is assumed \citep{balbus99}. The spatial averaging should be
taken on length scales no less than the local disc scale-height, and
the temporal average taken on timescales no less than the local
orbital period. These are, however, the relevant scales for vortex
formation via the RWI. Furthermore, our viscosity profile varies on
length-scales comparable to or even less than $H$ (e.g. the vertical transition
between high and low viscosity layers). Nevertheless, as a proof of
concept, our simulations do demonstrate %that viscous damping can
%significantly 
the importance of disc vertical structure on the RWI. That is, damping,
even confined to the disc atmosphere, can destroy Rossby vortices. 

Another drawback of a hydrodynamic viscous disc model is the
fact that it cannot mimic magneto-elliptic instabilities (MEI), which
are known to destroy vortices in magnetic discs
\citep{lyra11,mizerski12}. A natural question is how 
Rossby vortices are affected by the MEI when it only operates in
the disc atmosphere. Extension of the present work to global non-ideal
MHD simulations will be necessary to address RWI vortex formation in
layered discs.   

However, some improvements can be made within the viscous
framework. A static viscosity profile neglects the
back-reaction of the density field on the kinematic viscosity. Thus,
our simulations only consider how Rossby vortices 
respond to an externally applied viscous damping. A more
physical viscosity prescription should depend on the local column density
\citep{fleming03}, with viscosity decreasing with increasing column
density. The effective viscosity inside Rossby vortices would be
lowered relative to the background disk because disc vortices are
over-densities. If the over-density is large, then it is
conceivable that the vortex formation itself may render the effective
viscosity to be sufficiently low throughout the fluid column to allow
long term vortex survival.    
Preparation for this study is underway and results will be reported in
a follow-up paper. 
