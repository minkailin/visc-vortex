%\clearpage
\section{Summary and discussion}\label{summary}
We have performed customised hydrodynamic simulations of
non-axisymmetric instabilities in 3D viscous discs. We were motivated  
by the question of whether or not the Rossby wave instability, and
subsequent vortex formation, operates in layered accretion discs.  As
a first study, we adopted height-dependent kinematic viscosity
profiles, such that the disc midplane is of low viscosity ($\alpha\sim
10^{-4}$) and the disc atmosphere is of high viscosity ($\alpha\sim
10^{-2}$). 

We first considered viscous disc equilibria with a radial density
bump and varied the vertical dependence of viscosity. 
This experiment isolates the effect of viscosity on the
linear RWI. We found that the linear RWI is unaffected by viscosity,
layered or not. The viscous RWI remains dynamical and leads to vortex
formation on timescales of a few 10s of orbits. We continued these
simulations into the non-linear regime, but found that vortices became
stronger as the viscous layer is increased in thickness. We suggest this
counter-intuitive result is an artifact of the chosen viscosity
profile because it is radially structured: viscosity attempts to
restore the equilibrium radial density bump, which favours the
RWI. 

We also simulated vortex formation at plantary gap edges in layered
discs. In this case we expect viscosity to only act to damp the
instability. We find that introducing a viscous layer can inhibit long
term vortex formation. This occurs even when the vertical domain is 3
scale-heights and the viscous layer only occupies the uppermost
scale-height. We found that the non-axisymmetric energy densities have
weak vertical dependence, so the disturbance evolves
two-dimensionally. It appears that applying a large viscosity in the
disc atmosphere is sufficient to damp the instability throughout the
vertical column of the fluid.  

\subsection{Relation to other works}
%The work of \cite{pierens10} is most closely related to the present
%study.
\cite{pierens10} simulated the orbital migrtion of giant planets in
layered discs by prescribing a heigh-dependent viscosity profile. They
considered significant reduction in kinematic viscosity in going from
the disc atmosphere (the active zone, with $\alpha\sim10^{-2}$) to the
disc midplane (the dead zone, with $\alpha\sim10^{-7}$). According to
previous 2D simulations, such a low kinematic viscosity should lead to
the RWI \citep{valborro06,valborro07}. However, \citeauthor{pierens10}
did not report vortex formation (nor are vortices visible from their
plots). This result is consistent with our simulations. 

Another related study is that of \cite{oishi09}. 



\subsection{Caveats and outlooks}\label{caveats}
%We stress that the goal of the present study is to demonstrate the
%potential impact of disc vertical structure on Rossby vortices. We
%adopted simple 

{\bf viscosity does not mimic mri turbulence (need spatial and
  temporal averaging)}

{\bf no magneto-elliptic instability}

{\bf static viscosity profile. but viscosity should depend on
  density. enhanced density may cause lower viscosity so perhaps vortex would
  survive? implement dynamic viscosity} 

%The most significant caveat of our model is, by far, the use of 
%a kinematic viscosity to model accretion.   
