%\clearpage
\section{Summary and discussion}\label{summary}
We have performed customised hydrodynamic simulations of
non-axisymmetric instabilities in 3D viscous discs. We were motivated  
by the question of whether or not the Rossby wave instability, and
subsequent vortex formation, operates in layered accretion discs.  As
a first study, we adopted height-dependent kinematic viscosity
profiles, such that the disc midplane is of low viscosity ($\alpha\sim
10^{-4}$) and the disc atmosphere is of high viscosity ($\alpha\sim
10^{-2}$). 

We first considered viscous disc equilibria with a radial density
bump and varied the vertical dependence of viscosity. 
This experiment isolates the effect of viscosity on the
linear RWI. We found that the linear RWI is unaffected by viscosity,
layered or not. The viscous RWI remains dynamical and leads to vortex
formation on timescales of a few 10s of orbits. We continued these
simulations into the non-linear regime, but found that vortices became
stronger as the viscous layer is increased in thickness. We suggest this
counter-intuitive result is an artifact of the chosen viscosity
profile because it is radially structured: viscosity attempts to
restore the equilibrium radial density bump, which favours the
RWI. 

We also simulated vortex formation at plantary gap edges in layered
discs. In this case we expect viscosity to only act to damp the
instability. We find that introducing a viscous layer can inhibit long
term vortex formation. This occurs even when the vertical domain is 3
scale-heights and the viscous layer only occupies the uppermost
scale-height. We found that the non-axisymmetric energy densities have
weak vertical dependence, so the disturbance evolves
two-dimensionally. It appears that applying a large viscosity in the
disc atmosphere is sufficient to damp the instability throughout the
vertical column of the fluid. Our results suggest that vortex
formation at planetary gap edges require low viscosity
($\alpha\lesssim10^{-4}$) throughout the vertical extent of the
disc. 

\subsection{Relation to other works}
%The work of \cite{pierens10} is most closely related to the present
%study.
\cite{pierens10} simulated the orbital migrtion of giant planets in
layered discs by prescribing a heigh-dependent viscosity profile. They
considered significant reduction in kinematic viscosity in going from
the disc atmosphere (the active zone, with $\alpha\sim10^{-2}$) to the
disc midplane (the dead zone, with $\alpha\sim10^{-7}$). According to
previous 2D simulations, such a low kinematic viscosity should lead to
the RWI \citep{valborro06,valborro07}. However, \citeauthor{pierens10}
did not report vortex formation, nor are vortices visible from their
plots. This result is consistent with our simulations. 

\cite{oishi09} carried out MHD shearing box simulations with 
a resistivity profile that varied with height to model a
layered disc: the disc atmosphere was MHD turbulent while the disc
midplane remained stable against the MRI. They envisioned the active
zone as a vorticity source for vortex formation in the midplane.  
Although their setup is fundamentally different to ours, they also
reported a lack of coherent vortices in the dead zone. They argued
that the MHD turbulence in the acive layer was not sufficiently strong
to induce vortex formation in the dead zone. Our results suggest that,
if MHD turbulence can be represented by a viscosity, then tall
columnar vortices will be damped even if the turbulence is only
present in the disc atmosphere.    


\subsection{Caveats and outlooks}\label{caveats}
%We stress that the goal of the present study is to demonstrate the
%potential impact of disc vertical structure on Rossby vortices. We
%adopted simple 

The most important caveat of the current model is the viscous
prescription to mimic MRI turbulence. In doing so, an implicit
averaging is assumed \citep{balbus99}. The spatial avering should be
taken on length scales no less than the local disc scale-height, and
the temporal average taken on timescales no less than the local
orbital period. These are, however, the relevant scales for vortex
formation via the RWI. Furthermore, our viscosity profile varies on
length-scales comparable to or even less than $H$ (e.g. the transition
between high and low viscosity layers). Nevertheless, as a proof of
concept, our simulations do demonstrate the potential importance of
disc vertical structure on the RWI.   

Another drawback of a hydrodynamic viscous disc model is the
fact that it cannot mimic magneto-elliptic instabilities (MEI), which
are known to destroy vortices in magnetic discs
\citep{lyra11,mizerski12}. A natural question how 
Rossby vortices are affected by the MEI when it only operates in
the disc atmosphere. Extension of the present work to global non-ideal
MHD simulations will be neccessary to address RWI vortex formation in
layered discs.   

However, some improvements can be made within the viscous
framework. A static viscosity profile neglects the
back-reaction of the density field on the kinematic viscosity. Thus,
our simulations only consider how the RWI and Rossby vortices
respond to an externally applied axisymmetric viscous damping. A more
physical viscosity prescription should depend on the local column density
\citep{fleming03}, with viscosity decreasing with increasing column
density. The effective viscosity inside Rossby vortices would be
lowered relative to the background disk because disc vortices are
over-densities. If the over-density is large, then it is
conceivable that the vortex formation itself may render the effective
viscosity to be sufficiently low throughout the fluid column to allow
for long term survival.  
%radial boundary between high and low viscosity 
%(Of course, this is to say th) 
Preparation for this study is underway and results will be reported in
a follow-up paper. 
%{\bf static viscosity profile. but viscosity should depend on
%  density. enhanced density may cause lower viscosity so perhaps vortex would
%  survive? implement dynamic viscosity} 



%The most significant caveat of our model is, by far, the use of 
%a kinematic viscosity to model accretion.   
