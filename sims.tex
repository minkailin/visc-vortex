\section{Numerical experiments}\label{sims}
The necessary condition for the RWI --- a potential vorticity 
extremum \citep{li00} --- is either set as an initial condition
via a density bump, or obtained from a smooth disc by evolving it
under disc-planet interaction. The setup of each experiment is
detailed in subsequent sections.    

We adopt units such that $G=M_*=1$, and the reference radius $r_0=1$.    
We set $\sigma=0.5$ for the initial surface density profile and apply 
frictional damping within the shells $r<r_\mathrm{d,in} = 1.25\rin$
and $r>r_\mathrm{d,out}=0.84\rout$.

The fluid equations are evolved using the \pluto code \citep{mignone07} with 
the FARGO algorithm enabled  \citep{masset00,mignone12}. We employ a static
spherical grid with $(N_r, N_\theta, N_\phi)$ zones uniformly spaced
in all directions. For the present simulations the code was configured
with piece-wise linear reconstruction, a Roe solver and second order
Runge-Kutta time integration.   

Boundary conditions are imposed through ghost zones.   
Let the flow velocity parallel and normal to a boundary be
$v_\parallel$ and $v_\perp$, respectively. Two types of numerical 
conditions are considered for the $(r,\theta)$ boundaries:
\begin{inparaenum}[(a)]
\item \emph{reflective}: $\rho$ and $v_\parallel$ are symmetric with
  respect to the boundary while $v_\perp$ is anti-symmetric;  
\item \emph{unperturbed}: ghost zones retain their initial values
\end{inparaenum}. 
The boundary conditions adopted for all simulations is unperturbed in
$r$, reflective in $\theta$ and periodic in $\phi$. 


\subsection{Diagnostics}
We list several quantities calculated from simulation data for use in
results visualization and analysis.  
 
\subsubsection{Density perturbations}
The relative density perturbation $\delta\rho$ 
and the non-axisymmetric density fluctuation $\Delta\rho$ are defined as 
\begin{align}
  \delta\rho(\bm{r},t) \equiv \frac{\rho - \rho_i}{\rho_i}, \quad
  \Delta\rho(\bm{r},t) \equiv \frac{\rho -
    \aziavg{\rho}}{\aziavg{\rho}}, 
\end{align} 
where $\avg{\cdot}_\phi$ denotes an azimuthal average.  
In general $\Delta\rho$ accounts for the time evolution of 
the axisymmetric part of the density field, but if
$\p_t\avg{\rho}_\phi=0$ then $\Delta\rho$ is identical to
$\delta\rho-\avg{\delta\rho}_\phi$.    

\subsubsection{Vortical structures}
The Rossby number
\begin{align}
  Ro \equiv
  \frac{\hat{\bm{z}}\cdot\nabla\times\bv-\avg{\hat{\bm{z}}\cdot\nabla\times\bv}_\phi}{2\aziavg{\Omega}},   
\end{align} 
can be used to quantify the strength of vortical structures and to
visualize it. $Ro<0$ signifies anti-cyclonic motion with respect to
the background rotation. Note that while for thin discs 
the rotation profile is Keplerian, the shear is non-Keplerian for radially  
structured discs (i.e. $\Omega\simeq\Omega_k$ but the epicycle
frequency $\kappa\neq\Omega$). 

\subsubsection{Potential vorticity}
The potential vorticity (PV, or vortensity) is 
$  \bm{\eta}_\mathrm{3D} = \nabla\times~\bm{v}/\rho$. 
%Since we are considering geometrically thin discs, 
However, it will be convenient to work with vertically averaged
quantities. We define  
\begin{align}
  \eta_z = \frac{1}{\Sigma}\int \hat{\bm{z}}\cdot\nabla\times\bv dz
\end{align}
as the PV in this paper, where $\Sigma = \int\rho dz$ and the
integrals are confined to the computational domain. 
We recall for a 2D disc the vortensity is defined as 
$\eta_\mathrm{2D} \equiv \hat{\bm{z}}\cdot\nabla\times\bv/\Sigma$, and
extrema in $\eta_\mathrm{2D}$ is necessary for the RWI in
2D \citep{lovelace99,lin10}. If the velocity field is independent of 
$z$ then $\eta_z$ is proportional to $\eta_\mathrm{2D}$ (at fixed
cylindrical radius).    

\subsubsection{Perturbed kinetic energy density}  
We define the perturbed kinetic energy as
$W\equiv\rho|\bm{v}|^2/[\rho_i|\bm{v}(t~=0)|^2] - 1$, and its Fourier transform 
$W_m\equiv\int_0^{2\phi} W\exp{(-\mathrm{i}m\phi)}d\phi$. We will examine
$|W_m(r,\theta)|$ averaged over sub-portions of the $(r,\theta)$
plane. 
