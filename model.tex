\section{Governing equations}
We consider a three-dimensional, strictly isothermal, non-self-gravitating fluid disc orbiting a central
star of mass $M_*$. We use both spherical co-ordinates $\bm{r}=(r,\theta,\phi)$ and cylindrical co-ordinates 
$\bm{r}=(R, \phi, z)$ centred on the star and 
adopt a non-rotating frame. We also define $\psi \equiv \pi/2 -
\theta$ as the angular displacement from the disc midplane. For
convenience, we will sometimes refer to $\psi$ as the vertical
direction. The governing
equations are: 
\begin{align}
&\frac{\p\rho}{\p t} + \nabla\cdot\left(\rho\bm{v}\right) = 0,\label{cont_eq}\\
&\frac{\p\bm{v}}{\p t} + \bm{v}\cdot\nabla\bv = -\frac{1}{\rho}\nabla p -\nabla{\left(\Phi_*+\Phi_p\right)} 
 +\bm{f}_\nu + \bm{f}_d, 
\end{align}
where $\rho$ is the density, $\bv$ is the velocity field and 
$p=c_s^2\rho$ is the pressure. The isothermal sound speed is written
as $c_s = hr_0\Omega_k(r_0)$, where $h$ is the aspect-ratio at the reference radius $r_0$, 
$\Omega_k(R) = \sqrt{GM_*/R^3}$ is the Keplerian frequency and $G$ is the gravitational constant. 
$\Phi_*(r) = -GM_*/r $ is the stellar potential and $\Phi_p$ is a
planetary potential. Details of $\Phi_p$ are described in Section
\ref{planet}. 

Two dissipative terms are included in the momentu, equations: viscous
damping $\bm{f}_\nu$ and frictional damping $\bm{f}_d$. The viscous
force is 
\begin{align}
  \bm{f}_\nu = \frac{1}{\rho}\nabla\cdot\bm{T},
\end{align}
where 
\begin{align}
  \bm{T} = \rho\nu \left[\nabla\bv + \left(\nabla\bv\right)^\dagger
    - \frac{2}{3}\left(\nabla\cdot\bv\right)\bm{1} \right]
\end{align}
is the viscous stress tensor and $\nu$ is the kinematic viscosity 
(`$^\dagger$' denotes the transpose). The frictional force is
\begin{align}
  \bm{f}_d = -\gamma\left(\bv - \bv_\mathrm{ref}\right),
\end{align}
where $\gamma$ is the damping coefficient and the reference velocity field is taken 
to be the initial velocity field, $\bv_\mathrm{ref}=\bm{v}(t=0)$. %where $\init{\phantom{a}}$ denotes evaluation at $t=0$. 
$\nu$ and $\gamma$ are prescribed functions of position only, and are
specified later.   

\subsection{Disc model and initial conditions}\label{IC}
The numerical disc model occupies $r\in[\rin,\rout]$, $\theta\in[\theta_\mathrm{min},
  \pi/2]$ and $\phi\in[0,2\pi]$ in spherical co-ordinates. 
Only the upper disc is considered explicitly 
($\psi>0$), by assuming symmetry across the midplane. The maximum
angular height is $\psimax \equiv \pi/2 - \theta_\mathrm{min}$. 
The extent of the vertical domain is parametrized by $n_h\equiv
\tan{\psimax}/h$, i.e. the number of scale-heights at the reference 
radius.   

The disc is initially axisymmetric with zero cylindrical vertical
velocity: $\rho(t=0)\equiv\rho_i(R,z)$ and
$\bm{v}(t=0)\equiv(v_{Ri}\,, R\Omega_i\,, 0)$ in cylindrical
co-ordinates. The initial density field is set by assuming vertical
hydrostatic balance between gas pressure and stellar gravity:
\begin{align}\label{vert_balance}
  0 = \frac{1}{\rho_i}\frac{\p p_i}{\p z} + \frac{\p \Phi_*}{\p z},
\end{align}
where $p_i=c_s^2\rho_i$ is the initial pressure field. We write 
\begin{align}\label{init_den}
  \rho_i = \frac{\Sigma_i(R)}{\sqrt{2\pi}H(R)}
  \exp{\left\{\frac{1}{c_s^2}\left[\Phi_*(R) - \Phi_*(r)\right]\right\}},
\end{align}
where $H = c_s/\Omega_k$ is the pressure scale-height. The initial
surface density $\Sigma_i(R)$ is chosen as  
\begin{align}
  \Sigma_i(R) = \Sigma_0\left(\frac{R}{r_0}\right)^{-\sigma}\times B(R),
  %\left(1 - \sqrt{\frac{R_\mathrm{in}}{R + H_\mathrm{in}}}\right)\times B(R),
\end{align}
where $\sigma$ is the power-law index and the surface density scale $\Sigma_0$ 
is arbitrary for a non-self-gravitating disc. 
The bump function $B(R)$ is
\begin{align}
B(R) = 1 + \left(A - 1\right)\exp{\left[-\frac{(R-r_0)^2}{2\Delta R^2}\right]},
\end{align}
where $A$ is the bump amplitude and $\Delta R$ is the bump width. The initial surface 
density has bump if $A>1$ and is smooth if $A=1$. 

The initial angular velocity is chosen
to 
satisfy centrifugal balance with pressure and stellar gravity: 
\begin{align}\label{init_vphi}
  R\Omega^2_i = \frac{1}{\rho_i}\frac{\p p_i}{\p
    R} + \frac{\p\Phi_*}{\p R},  
\end{align}
so $\Omega_i=\Omega_i(R)$ for a strictly isothermal equation of
state. %% Note that Eq. \ref{init_den} and Eq. \ref{init_vphi} are the
%% vertical and cylindrical radial momentum equations assuming a steady
%% state and negligible viscosity. 

The initial cylindrical radial velocity $v_{Ri}$ and the viscosity
profile $\nu$ depends on the numerical experiment, and will be
described along with simulation results. Note that $v_{Ri}$ and $\nu$
are not independent if one additionally requires a steady-state, as
discussed below (see also \S\ref{density_bump}).   

\subsubsection{Viscous equilibria for a radially structured
  disc}\label{visc_eq} 
The RWI is a linear instability. That is, non-axisymmetric
disturbances grow exponentially with respect to a time-independent
axisymmetric background. A convenient way to simulate the RWI is to
initialize an inviscid disc with a density bump \citep{li00}. 

%%Complications arise when trying to extend RWI simulations 
However, in a viscous disc, a density bump will undergoe viscous
spreading \citep{lyndenbell74}, meaning a linear instability cannot be
defined in the usual way. 
%% The viscous time-scale
%% associated with a density bump of characteristic width $H$ is 
%% $t_\nu=H^2/\nu = O(10P_0)$ for the maximum viscosity values
%% we consider. This is still longer than the expected RWI  
%% linear growth time-scale, $t_\mathrm{RWI}=O(P_0)$, but $t_\nu$ can be
%% shorter than our simulation time-scales,
%% $t_\mathrm{sim}=O(100P_0)$. 
Therefore, if a density bump is set as initial conditions for the RWI
in a viscous disc, they should correspond to a
steady-state. 
%Otherwise, the density bump will undergo both
%axisymmetric viscous evolution and the non-axistymmetric instability.   

In choosing $\rho_i$ and $\Omega_i$, we have neglected radial
velocities and viscous forces in the steady-state vertical and
cylindrical radial momentum equations (Eq. \ref{vert_balance} and
Eq. \ref{init_vphi}, respectively). This is standard practice for
accretion disc modelling \citep[e.g.][]{takeuchi02}.   

However, $v_{Ri}$ and $\nu$ cannot be ignored in the azimuthal
momentum equation. Indeed, if a stead-state is desired, then the
conservation of angular momentum in a viscous disc implies special
relations between the viscosity, cylindrical radial velocity and
density field. We do this for simulations in \S\ref{density_bump}.   


\subsection{Damping}
Frictional damping is often used in hydrodynamical simulations in order
to reduce reflections from boundaries
\citep[e.g.][]{bate02,valborro07}. We employ it for this  
purpose in the radial direction. The damping coefficient $\gamma$ is only 
non-zero within the `damping zones', here taken to be $r\leq \dampin,\, 
r\geq \dampout$, 
\begin{align}
  \gamma = \hat{\gamma}\Omega_i\times
  \begin{cases}
    \xi_\mathrm{in}(r) & r\leq\dampin \\
    \xi_\mathrm{out}(r) & r\geq\dampout \\
  \end{cases},
\end{align}
where $\hat{\gamma}$ is the dimensionless damping rate. We choose
\begin{align}
\xi_\mathrm{in}(r) = \left(\frac{\dampin - r}{\dampin - \rin}\right)^2 \text{ and } \quad
\xi_\mathrm{out}(r) = \left(\frac{r - \dampout}{\rout- \dampout}\right)^2
\end{align}
for the inner and outer radial zones, respectively. 

\subsection{Planet potential}\label{planet}
Our disc model has the option to include a planet potential $\Phi_p$,
\begin{align}
  \Phi_p(\bm{r},t) = -\frac{GM_p}{\sqrt{|\bm{r}-\bm{r}_p(t)|^2 + \epsilon^2_p}},
\end{align}
where $M_p$ is the planet mass,
$\bm{r}_p(t)~=~(r_0,\,\pi/2,\,\Omega_k(r_0)t+\pi)$ its position in
spherical co-ordinates, $\epsilon_p = \epsilon_{p0}r_h$ is a
softening length, and $r_h=(M_p/3M_*)^{1/3}r_0$ is the Hill radius. 
For the purpose of our study $\Phi_p$ is considered as a 
fixed external potential. That is, orbital migration is neglected.  
