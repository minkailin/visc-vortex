\section{Additional simulations with artificial density bumps}
%% According to steady-state requirements described in \S\ref{visc_eq},
%% Eq. \ref{visc_profile} is not an ideal model for viscosity, because
%% it leads to a $R$-depdendent mass flux.  
%% %% However, this dependency varies
%% %% on a global scale, whereas the instability of concern is radially
%% %% local.
%% An ideal choice for the kinematic viscosity
%% profile is 
%% \begin{align}\label{nu_ideal}
%%   \hat{\nu}_\mathrm{ideal} =
%%   \hat{\nu}_0\left[1+Q(z)\right]\frac{\rho_i(r_0,z)}{\rho_i(R,z)}
%%   \frac{\left.d\ln{\Omega}/d\ln{R}\right|_{r_0}}{d\ln{\Omega}/d\ln{R}}.           
%% \end{align}
%% This corresponds to viscous layers parallel to the midplane. 
%% %% In a numerical simulation, maintaining a steady-state with
%% %% Eq. \ref{nu_ideal} is still dependent on boundary conditions. 
%% %% For example, a reflective boundary at $\theta=\theta_\mathrm{min}$
%% %% would  result in mass build-up at the location where the
%% %% transition to a viscous layer, say $z=z_\nu$, intersects the
%% %% spherical grid at $\theta=\theta_\mathrm{min}$. 


%% We repeat several simulations with the viscosity profile given by
%% Eq. \ref{nu_ideal}. Specifically, we consider simulations initialized
%% with a density bump. 
